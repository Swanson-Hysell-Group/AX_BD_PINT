\documentclass[draft]{agujournal2019}
\usepackage{url} %this package should fix any errors with URLs in refs.
\usepackage{lineno}
\usepackage[inline]{trackchanges} %for better track changes. finalnew option will compile document with changes incorporated.
\usepackage{soul}
\linenumbers

\draftfalse

%% Enter journal name below.
%% Choose from this list of Journals:
%
% JGR: Atmospheres
% JGR: Biogeosciences
% JGR: Earth Surface
% JGR: Oceans
% JGR: Planets
% JGR: Solid Earth
% JGR: Space Physics
% Global Biogeochemical Cycles
% Geophysical Research Letters
% Paleoceanography and Paleoclimatology
% Radio Science
% Reviews of Geophysics
% Tectonics
% Space Weather
% Water Resources Research
% Geochemistry, Geophysics, Geosystems
% Journal of Advances in Modeling Earth Systems (JAMES)
% Earth's Future
% Earth and Space Science
% Geohealth
%


\journalname{JGR: Solid Earth}


\begin{document}

\title{High geomagnetic field intensity in the late Mesoproterozoic recorded by Midcontinent Rift anorthosite xenoliths}


\authors{Yiming Zhang\affil{1}, Nicholas L. Swanson-Hysell\affil{1}, James D. Miller Jr.\affil{2}, Margaret S. Avery\affil{1,3}}

\affiliation{1}{Department of Earth and Planetary Science, University of California, Berkeley, CA, USA}
\affiliation{2}{Department of Earth and Environmental Sciences, University of Minnesota, Duluth, MN, USA}
\affiliation{3}{Geology, Minerals, Energy, and Geophysics Science Center, U.S. Geological Survey, Moffett Field, CA, USA}
\correspondingauthor{Yiming Zhang}{yimingzhang@berkeley.edu}


\begin{keypoints}
\item New ca. 1092 Ma Midcontinent Rift anorthosite record high paleointensity values
\item High VADM of the geomagnetic field during the late Mesoproterozoic is inconsistent with previous interpretation of long term decay
\item A strong geodynamo like today likely persisted throughout 25 myr during the Midcontinent Rift development
\end{keypoints}


% keywords: absolute paleointensity, Laurentia, Midcontinent Rift, anorthosite, 



\begin{abstract}
High-quality absolute paleointensity 

\end{abstract}
Retrieving paleomagnetic information from Precambrian rocks can aid in our understanding of the long-term evolution of the geodynamo. A recent paleointensity study by \cite{Sprain2018a} reported a ca. 1.1 Ga paleomagnetic field strength similar to today based on the volcanics of the Midcontinent Rift System (MRS). That study suggested that this result is consistent with models wherein this time period of the Proterozoic is characterized by a strong rather than weak geomagnetic field. However, to further evaluating the evolution of the geomagnetic field during the Precambrian, more paleointensity data are needed. 

%The MRS emplaced a vast amount of extrusive and intrusive rocks spanning a prolonged period of $\sim$25 Myrs. \cite{Swanson-Hysell2019a} published an extensive compilation of the MRS volcanics with paired high-precision CA-ID-TIMS U-Pb geochronology and high-quality paleomagnetic directions, revealing at least two dipole reversals and rapid differential plate motion of the ancient North American plate, Laurentia, during the emplacement of MRS. The synthesized North American apparent polar wander path (APWP) was interpreted to be composed primarily of rapid plate motion \citep{Swanson-Hysell2019a}. Through comparison between paleomagnetic pole positions and the APWP, such rapid plate motion and high-precision paleomagnetic and geochronological framework revealed by the MRS volcanics can provide insights into age constraints on units with poor or no geochronology data. This framework allows for further paleomagnetic contributions beyond the Midcontinent Rift volcanics. 

  The hypabyssal intrusions of the ophitic Beaver River diabase (BRD) and the Ca-rich anorthosite xenoliths that it hosts are the targets of this study. The diabase contains abundant Fe-Ti oxides which are dominant carriers of paleomagnetic information. The nearly-pure anorthosite xenoliths have potential for paleomagnetic investigation because plagioclase hosts can protect the exsolved magnetic minerals from post-formation alteration \citep{Tarduno2005a}, and thus may faithfully record paleointensity information at the time of formation. A typical outcrop of anorthosite xenolith residing in the diabase and a hand sample of the anorthosite are shown in Fig. \ref{fig:field}(A, B). Preliminary petrographic analyses show that the anorthosites are dominantly monomineralic with recrystallization textures (Fig. \ref{fig:field}C). Crosscutting relationships together with high-precision geochronology dates from \cite{Swanson-Hysell2019a} bracket the age of the BRD to be between 1091.61 $\pm$ 0.14 Ma  (the age of the younger Silver Bay intrusions) and 1093.94 $\pm$ 0.28 Ma  (the age of the Palisade rhyolite). 
  
\begin{figure}
\noindent\includegraphics[width=\textwidth]{Field_photo_petro_photo.pdf}
\caption{\small{(A) A typical outcrop of an anorthosite xenolith residing in the Beaver River diabase. (B) A hand sample of anorthosite. The large reflective face at the bottom center is a cleavage plane of a centimeter-size plagioclase crystal. The scale is 9 cm in total. (C) Cross-polarized light image of an anorthosite thin section. The second order birefringence of plagioclase indicates a high Ca content. The closely packed plagioclase crystals show recrystallization texture. (D) High-magnification SEM image of Fe-Ti oxides exsolved from a pyroxene crystal included in a plagioclase crystal. (E) Large, interstitial magnetite-ilmenite intergrowths between two plagioclase crystals. (F) A large Fe-Ti oxide grain with magnetite-ilmenite intergrowths next to a pyroxene in diabase. an: anorthite; px: pyroxene; mag: magnetite; ilm: ilmenite}}
\label{fig:field}
\end{figure}

  Most thermally demagnetized anorthosites and alternating field (AF) demagnetized diabase specimens have minimal secondary components and their magnetization is interpreted to be dominated by a primary thermal remanent magnetization. Characteristic magnetizations from both lithologies yield indistinguishable site-mean directions and the virtual geomagnetic poles (VGPs) fall close to the 1095 Ma pole from the synthesized APWP of \cite{Swanson-Hysell2019a}, which is in agreement with the geochronological constraints  (Fig. \ref{fig:pmag}). 

I conducted a comparative study of modified IZZI paleointensity experiments on both the diabase and anorthosite, with a group treated with low-field AF demagnetization after each in-field step and another group without. Paleointensity experiments for most diabase specimens result in non-ideal Arai plots often with poor pTRM checks, making them difficult to interpret for paleointensity estimate (Fig. \ref{fig:pmag}). These results can be interpreted to be associated with observed color change of the diabase after heating, which suggests alterations of the oxides. The experiments conducted on anorthosite samples, on the other hand, had a high success rate of $>$ 50\%. Moreover, the resulting success rate for the anorthosite group with the AF treatment is even higher. The more straight Arai plots are likely due to the AF steps mitigating the exhibition of pTRM tails often associated with non-ideal behavior of multi-domain grains that are demagnetized by the pre-treatment step. However, as the nearly pure anorthosite samples did not display dramatic alterations that are easily detectable with common petrographic techniques, a question rises as why some of the compositionally similar anorthosites passed the paleointensity selection results and some did not. 

\begin{figure}
\noindent\includegraphics[width=\textwidth]{pmag_plot_.pdf}
\caption{\small{(A) Site-mean directions of anorthosite and diabase plotted on an equal-area plot. (B) Calculated VGPs from the anorthosite and diabase plotted in context of a previously synthesized ca. 1.1 Ga Laurentia APWP from the Midcontinent Rift volcanics \citep{Swanson-Hysell2019a}. VGPs from both lithologies fall close to the 1095 Ma pole position. (C) Example Arai plots for diabase. Most diabase and many anorthosite specimens were rejected by selection criteria due to the zigzagging behavior shown in the diabase example plot. (D) Example Arai plots for anorthosite. A typical anorthosite specimen that passes selection shows straight Arai plot with a high paleointensity estimate before cooling rate correction. Note both specimens in (C) and (D) show dominantly single component magnetization in the inset orthogonal plots. The estimated field intensity is not cooling rate corrected.}}
\label{fig:pmag}
\end{figure}

At the IRM, I was seeking the answer to this question with the help of the vibrating sample magnetometer (VSM) systems, including a newly installed Lake Shore VSM which greatly helped improve measurement resolution on samples with weak magnetizations (Fig. \ref{fig:backfield}). Backfield demagnetization experiments were conducted and used to develop coercivity spectra (Fig. \ref{fig:backfield}). The spectra were subsequently modeled to fit for the distributions of different  populations of magnetic particles using similar procedure as \cite{Maxbauer2016a}. Most spectra can be well approximated with models with one component or two components with overlapping coercivity ranges (Fig. \ref{fig:backfield}). The dominance of single-component distributions from the unmixed coercivity spectra likely suggests minimal alterations or formation of secondary magnetic mineral within the samples. By compiling the values of median destructive field (MDF) from all measured specimens and categorizing them in terms of their sister specimens' paleointensity results, I found that the anorthosites samples that pass paleointensity selection criteria have distinctively higher MDF values than those did not (Fig. \ref{fig:backfield}), and all diabase specimens have low MDF values like the non-ideal anorthosites (Fig. \ref{fig:backfield}). The higher MDFs ($\sim$60 mT) of the anorthosites are similar to what is typical of stoichiometric, single-domain magnetite grains which are favored for paleointensity experiments. Furthermore, the low MDFs displayed in other specimens could be linked to a dominant population of large interstitial Fe-Ti oxides that are more likely to form multi-domain grains, which have been observed in both the anorthosite and diabase (Fig. \ref{fig:field}). In addition, preliminary tests on single anorthosite crystals might suggest that not all single crystals are necessarily better for paleointensity research when comparing to bulk samples, given their similar low MDF to the failed diabase and anorthosite specimens. However, more single crystal grains with paired paleointensity data and rock magnetic data may be needed before a general conclusion can be drawn. 

Taken together, the rock magnetism experiment results support my paleointensity selection criteria in that rock specimens that produce straight Arai plots and have MDF similar to stoichiometric single-domain magnetite are preferentially selected. The preliminary paleointensity results yielded consistent estimates. The cooling rate corrected site-mean paleointensity estimate is about 40 $\mu$T, consistent with results from \cite{Sprain2018a} that the Earth's magnetic field strength at surface in the late Mesoproterozoic was close to that of today. 

\begin{figure}
\begin{center}
	\noindent\includegraphics[width=0.5\textwidth]{backfield_example_.pdf}
\end{center}
\caption{\small{(A) Backfield measurement data acquired using a Princeton VSM on an anorthosite specimen. The noise level on coercivity plot is high and a heavy smoothing is needed to make subsequent coercivity modeling interpretable. (B) Backfield measurement data acquired using a Lake Shore VSM on the same specimen as that in (A). The inset coercivity unmixing plot shows that the much smoother initial backfield curve resulted in a tight single-component model fit. (C) Box plot of distribution of the median destructive field of all measured specimens. The anorthosite specimens that pass selection criteria have distinctly higher median destructive field (MDF) than other groups.}}
\label{fig:backfield}
\end{figure}








\section*{Introduction}

\section*{Method}

\section*{Results}

\section*{Discussion}

\section*{Conclusion}

\acknowledgments
I thank Dario Bilardello, Peat Solheid, Mike Jackson, Josh Feinberg, and Bruce Moskowitz for their tremendous help of instrumental operations, data interpretations, and research guidance. I thank the IRM for the generous U.S. Visiting Student Fellowship. I thank Nicholas Swanson-Hysell at UC Berkeley for advising this research. Margaret Avery provided field assistance.


\bibliography{}


\end{document}



More Information and Advice:




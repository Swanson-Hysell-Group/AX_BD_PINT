\documentclass[draft]{agujournal2019}
\usepackage{url} %this package should fix any errors with URLs in refs.
\usepackage{lineno}
\usepackage[inline]{trackchanges} %for better track changes. finalnew option will compile document with changes incorporated.
\usepackage{soul}
\linenumbers

\draftfalse

%% Enter journal name below.
%% Choose from this list of Journals:
%
% JGR: Atmospheres
% JGR: Biogeosciences
% JGR: Earth Surface
% JGR: Oceans
% JGR: Planets
% JGR: Solid Earth
% JGR: Space Physics
% Global Biogeochemical Cycles
% Geophysical Research Letters
% Paleoceanography and Paleoclimatology
% Radio Science
% Reviews of Geophysics
% Tectonics
% Space Weather
% Water Resources Research
% Geochemistry, Geophysics, Geosystems
% Journal of Advances in Modeling Earth Systems (JAMES)
% Earth's Future
% Earth and Space Science
% Geohealth
%


\journalname{JGR: Solid Earth}


\begin{document}

\title{High geomagnetic field intensity in the late Mesoproterozoic recorded by Midcontinent Rift anorthosite xenoliths}


\authors{Yiming Zhang\affil{1}, Nicholas L. Swanson-Hysell\affil{1}, James D. Miller Jr.\affil{2}, Margaret S. Avery\affil{1,3}}

\affiliation{1}{Department of Earth and Planetary Science, University of California, Berkeley, CA, USA}
\affiliation{2}{Department of Earth and Environmental Sciences, University of Minnesota, Duluth, MN, USA}
\affiliation{3}{Geology, Minerals, Energy, and Geophysics Science Center, U.S. Geological Survey, Moffett Field, CA, USA}
\correspondingauthor{Yiming Zhang}{yimingzhang@berkeley.edu}


\begin{keypoints}
\item New ca. 1092 Ma Midcontinent Rift anorthosite record high paleointensity values
\item High VADM of the geomagnetic field during the late Mesoproterozoic is inconsistent with previous interpretation of long term decay
\item A strong geodynamo like today likely persisted for about 25 myr during the Midcontinent Rift development
\end{keypoints}


% keywords: absolute paleointensity, Laurentia, Midcontinent Rift, anorthosite, geodynamo



\begin{abstract}
New high-quality absolute paleointensity data from anorthosite xenoliths within the Beaver River diabase intrusions of the North American Midcontinent Rift (MCR) record a virtual axial dipole moment (VADM) estimate of $\sim$80 ZAm$^2$ ca. 1092 Ma. This cooling rate corrected paleointensity estimate from an MCR intrusive unit is similar to previous results from extrusive units that span from ca 1108 Ma to ca. 1085 Ma. Taken together, new and previously published paleointensity data from the Midcontinent Rift support that a geomagnetic field strength persisted through the Midcontinent Rift development and the data are inconsistent with the interpretation of a decay of Earth's geodynamo strength through the late Mesoproterozoic. 


\end{abstract}

\section*{Introduction}
%The question of the core and the importance of the observational data as complementary for numerical simulation work.

That we have a solid inner core today is the result of secular cooling and differentiation of the layered Earth. The timing of inner core formation during Earth's thermal evolution is amongst the major unsolved questions in Earth's history. Due to the uncertain constraints on the thermal conductivity of the core \cite{Konopkova2016a, Gomi2013a, Ohta2016a}, numerical simulations have given drastically different predictions on the inner core nucleation age, ranging from about half a billion years ago to prior to the Archean \cite{Pozzo2012a, Gubbins2004a, Nimmo2015a}. Modeling has suggested that the emergence of the solid inner core could have expressed as changes of Earth's surface magnetic field strength due to the onset of compositional convection between the solid inner core and the liquid outer core \cite<e.g.>{Aubert2009a}. Thus, absolute paleointensity records preserved by magnetic minerals in igneous rocks may provide us with a probe into the birth of the inner core. Acquiring these observational data is crucial for advancing our understanding of Earth's thermal evolution and for testing existing models on the commencement of the inner core growth. However, due to the scarcity of well-preserved ancient rocks and low success rates typical of paleointensity experiments, only 9\% of the records in the current global paleointensity database (\href{http://earth.liv.ac.uk/pint/}{PINT, \cite{Biggin2010a}}) are from before the Cambrian and the existing records sparsely span over 3 billion years of time (Fig. \ref{fig:all_PINT_data}). 

\citeA{Bono2019a} filtered the available Precambrian paleointensity database (PINT) based on a set of quality criteria. A polynomial fit through the resulting paleointensity compilation from the Archean and the Proterozoic was used to interpret that a decrease trend of the estimated geomagnetic dipole strength occurred before the inner core started to form. In addition, \citeA{Bono2019a} recovered very low paleointensity values from ca. 565 Ma single silicate crystals  that were interpreted to be consistent with it reflecting the weak geodynamo after progressive cooling of the core before the onset of the rigorous compositional convection driven by the nucleation of the solid inner core. Such decay trend is recently supported by new low paleointensity estimates derived from the ca. 720 Ma Franklin Large Igneous Province \cite{Lloyd2021a}. 

However, previous paleointensity data from the MCR rock units \cite{Pesonen1983a, Kulakov2013a, Sprain2018a} record VADM values distinctly higher than what is predicted by the long term decay curve. These rock units notably are extrusive volcanics and span a tme period of about 25 myr during the development of the Midcontinent Rift magmatism. Many previous results from the Midcontinent Rift show non-ideal double-slope paleointensity behavior, which are difficult to interpret as they can yield two paleointensity estimates: a higher paleointensity estimate from the low-temperature portion and a lower paleointensity estimate from the high-temperature portion. Nevertheless, single-slope, ideal behavior from the Lake Shore Traps \cite{Kulakov2013a} lava flows passed the selection criteria of \citeA{Bono2019a}, and a few lava flows yielded single-slope behavior from the Osler Volcanics \cite{Sprain2018a}. These data are inconsistent with the interpretation of a low geomagnetic field strength during the Midcontinent Rift because they yield significantly higher VADM estimates than that projected by the trend from \citeA{Bono2019a}. 

To further enrich the Precambrian paleointensity database, and to test the hypothesis that the Earth's geomagnetic dipole strength remained at high values throughout the Midcontinent Rift, in this study we perform ``IZZI" paleointensity experiments \cite{Tauxe2004a} on a suite of anorthosite xenoliths hosted by the intrusive Beaver River diabase of the Beaver Bay Complex within the Midcontinent Rift. 


%Retrieving paleomagnetic information from Precambrian rocks can aid in our understanding of the long-term evolution of the geodynamo. A recent paleointensity study by \cite{Sprain2018a} reported a ca. 1.1 Ga paleomagnetic field strength similar to today based on the volcanics of the Midcontinent Rift System (MRS). That study suggested that this result is consistent with models wherein this time period of the Proterozoic is characterized by a strong rather than weak geomagnetic field. However, to further evaluating the evolution of the geomagnetic field during the Precambrian, more paleointensity data are needed. 

%The MRS emplaced a vast amount of extrusive and intrusive rocks spanning a prolonged period of $\sim$25 Myrs. \cite{Swanson-Hysell2019a} published an extensive compilation of the MRS volcanics with paired high-precision CA-ID-TIMS U-Pb geochronology and high-quality paleomagnetic directions, revealing at least two dipole reversals and rapid differential plate motion of the ancient North American plate, Laurentia, during the emplacement of MRS. The synthesized North American apparent polar wander path (APWP) was interpreted to be composed primarily of rapid plate motion \citep{Swanson-Hysell2019a}. Through comparison between paleomagnetic pole positions and the APWP, such rapid plate motion and high-precision paleomagnetic and geochronological framework revealed by the MRS volcanics can provide insights into age constraints on units with poor or no geochronology data. This framework allows for further paleomagnetic contributions beyond the Midcontinent Rift volcanics. 

\section*{Geologic Settings}
The North American Midcontinent Rift (MCR) is a failed intracontinental rift where protracted magmatic activity lasted from \textit{ca.} 1109 Ma to \textit{ca.} 1084 Ma \cite{Swanson-Hysell2019a}. Midcontinent Rift rocks extensively outcrop in today's Lake Superior region, with the total extent traceable by arcuate magnetic and gravity anomalies that extend to the southwest to Kansas, and to the southeast, to southern Michigan \cite{Hinze2020a}. Previous studies have divided magmatic activity in the rift into four stages based on interpreted changes in relative magmatic volume and the nature of magmatism: early ($\sim$1109–1104 Ma), latent ($\sim$1104–1098 Ma), main ($\sim$1098–1090 Ma) and late ($\sim$1090–1083 Ma) \cite{Vervoort2007a, Heaman2007a, Miller2013a}. In northeastern Minnesota, the Early Gabbro Series and the Felsic Series rocks of the Duluth Complex and reversed-polarity lavas of the lower North Shore Volcanic Group were emplaced during the early stage. The more voluminous Duluth Complex Layered Series and the plagioclase-rich Anorthositic Series, together with an associated $\sim$8 km thick extrusive volcanic sequences of the North Shore Volcanic Group (NSVG), were rapidly emplaced about 10 myr later at \textit{ca.} 1096 Ma during the main stage \cite{Paces1993a, Swanson-Hysell2020a}. 

The Beaver Bay Complex, which sits stratigraphically above the Duluth Complex, is another intrusive complex that resulted from main stage magmatism. The exposed area of the Beaver Bay Complex is $\sim$1000 km\textsuperscript{2} where it has been mapped along the northwestern shore of Lake Superior in northeastern Minnesota (Fig. \ref{fig:Geologic_map}). The Beaver Bay Complex is a multi-phase, composite intrusive complex that intrudes parts of the NSVG (Fig. \ref{fig:Geologic_map}; \citeA{Miller1997a, Swanson-Hysell2020a}). Distinct from the deep plutonic intrusions of the Duluth Complex, the majority of the Beaver Bay Complex is formed of hypabyssal intrusions that were emplaced as dikes and sills at shallow depths \cite{Miller1997a}. Most of the Beaver Bay Complex intrusions are dioritic to gabbroic in composition \cite{Miller1997a}. The main lithology of the Beaver River diabase dikes and sills network within the Beaver Bay Complex is an ophitic olivine gabbro (Fig. \ref{fig:Field_photo}), but in wider areas of dikes and the upper parts of thick sills, this rock type can abruptly transition into intergranular olivine oxide gabbro, then into subprismatic (and commonly foliated) ferrogabbro, and finally into granophyric monzodiorite. The more evolved and later emplaced components of the Beaver River diabase network are commonly distinguished as the Silver Bay intrusions in the southern Beaver Bay Complex (Fig. \ref{fig:Geologic_map}). Overall being intermediate in composition, the Silver Bay intrusions lithologies range from ophitic olivine gabbro to ferrogranite \cite{Shank1989a}. Field mapping by \citeA{Miller1994a} found intrusive relationships between the Silver Bay intrusions and the Beaver River diabase. Angular inclusions of the host Beaver River diabase within marginal zones of the Silver Bay intrusions led \citeA{Miller1997a} to interpret that the Silver Bay intrusions intruded after the diabase crystallized.

One distinctive feature of the Beaver River diabase is its inclusions of anorthosite xenoliths. In the southern part of the Beaver Bay Complex, the Beaver River diabase occurs as dikes and sills, typically including anorthosites with various sizes ranging from centimeters to over 150 meters (Figs. \ref{fig:Geologic_map}, \ref{fig:Field_photo}; \citeA{Grout1939a, Morrison1983a}). The diabase in this region intrudes the Palisade rhyolite of the North Shore Volcanic Group (Fig. \ref{fig:Geologic_map}), which has a $^{206}$Pb/$^{238}$U date of 1093.94 $\pm$ 0.28 Ma (2$\sigma$ analytical uncertainty is presented for CA-ID-TIMS dates throughout this work; \citeA{Swanson-Hysell2019a}). The Beaver River diabase is locally intruded by the Silver Bay intrusions (Fig. \ref{fig:Geologic_map}). An aplite unit within the granophyre zone of one of these Silver Bay intrusions has a $^{206}$Pb/$^{238}$U date of 1091.61 $\pm$ 0.14 Ma \cite{Swanson-Hysell2019a}. Another arcuate, sill-like diabase body mapped as the Beaver River diabase outcrops along the eastern part of the complex (Fig. \ref{fig:Geologic_map}; \citeA{Miller1997a}). The diabase composition there is similar to that in the south and it also contains large anorthosite xenoliths with dimensions that exceed 100 meters at Carlton Peak (Fig. \ref{fig:Geologic_map}). The Beaver River diabase in the northern part of the complex, near the Houghtaling Creek area, typically forms narrow, near-vertical dikes instead of sheets in the southern and eastern regions (Fig. \ref{fig:Geologic_map}; \citeA{Miller1994a}). The diabase in this region only locally contains xenoliths of anorthosite. 

Hundreds of anorthosite xenoliths have been recognized and mapped within the Beaver River diabase (Fig. \ref{fig:Geologic_map}). Many hill tops in the Beaver Bay Complex, such as Carlton Peak and Britton Peak, are large anorthosite blocks (which lead \citeA{Lawson1893a} to erroneously conclude that they were relict Archean topography). Later work established the anorthosite blocks as xenoliths, which are now extensively documented through geologic mapping of the region (Fig. \ref{fig:Geologic_map}; \citeA{Miller2001a, Miller1988a, Miller1989a, Boerboom2004a, Boerboom2006a, Boerboom2006b, Boerboom2007a}) and outcrop-scale exposures (Fig. \ref{fig:Field_photo}). In the field, the anorthosites typically appear as subrounded to rounded, light-colored, translucent blocks that are in sharp contact with the hosting diabase (Fig. \ref{fig:Field_photo}). They also occur as exposures whose contact with the diabase is covered (Fig. \ref{fig:Field_photo}). \citeA{Grout1939a} suggested that the rounded anorthosites are the result of abrasion during transportation as they were entrained by the diabase (i.e. physical weathering within a magmatic system). While the Beaver River diabase is chilled against the North Shore Volcanic Group lithologies that it intrudes, the diabase is not chilled against the margin of the anorthosite xenoliths \cite{Morrison1983a, Miller1997a}. The lack of chilled contacts is consistent with the anorthosite being at elevated temperatures and cooling at the same time as the diabase magma (Fig. \ref{fig:thermal_history_model}).

The anorthosite xenoliths are dominantly monomineralic plagioclase that has an average anorthite content of $\sim$70\% \cite{Morrison1983a, Doyle2016a}. Interstitial pyroxene and olivine are present in minor concentrations in the xenoliths. Within the Carlton Peak anorthosite xenolith, up to 10 cm oikocrysts of olivine and pyroxene can occur. Nevertheless, the overall olivine content in the anorthosites is low. Interstitial titanomagnetite-ilmenite intergrowths that exceed 100 $\mu$m can be found with microscopy and $<$20 $\mu$m Fe-Ti oxide grains can be detected with scanning electron microscopy (Fig. \ref{fig:Field_photo}). Based on textural differences \citeA{Morrison1983a} divided the anorthosite xenoliths into four groups: one group which typically have well-developed granoblastic texture characterized by equigranular plagioclase crystals; another group which have interlocking, lath-shaped plagioclase crystals; an intermediate group which can have both granoblastic texture and interlocking plagioclase laths; and a brecciated group that have brittle deformation textures superposed on pre-existing textures. 

The hypabyssal intrusions of the ophitic Beaver River diabase (BRD) and the Ca-rich anorthosite xenoliths that it hosts are the targets of this study. The diabase contains abundant Fe-Ti oxides which are dominant carriers of paleomagnetic information. The nearly-pure anorthosite xenoliths have potential for paleomagnetic investigation because plagioclase hosts can protect the exsolved magnetic minerals from post-formation alteration \citep{Tarduno2005a}, and thus may faithfully record paleointensity information at the time of formation. A typical outcrop of anorthosite xenolith residing in the diabase and a hand sample of the anorthosite are shown in Fig. \ref{fig:field}(A, B). Preliminary petrographic analyses show that the anorthosites are dominantly monomineralic with recrystallization textures (Fig. \ref{fig:field}C). Crosscutting relationships together with high-precision geochronology dates from \cite{Swanson-Hysell2019a} bracket the age of the BRD to be between 1091.61 $\pm$ 0.14 Ma  (the age of the younger Silver Bay intrusions) and 1093.94 $\pm$ 0.28 Ma  (the age of the Palisade rhyolite). 

High-precision $^{206}$Pb/$^{238}$U zircon date of 1091.83 $\pm$ 0.21 from one anorthosite in Silver Bay area tightly constraint the age of the diabase intrusion to be 1091.7 $\pm$ 0.2 Ma. previous paleomagnetic directional data from \citeA{Zhang2021a} reveal that vast majority of the anorthosite xenoliths have minimal secondary remanence and single component characteristic remanence magnetizations carried by low-Ti magnetite. Step-wise demagnetization experiments show that the characteristic remanence often unblock sharply between temperature steps 500\textdegree C and 580\textdegree C. Moreover, the site mean paleodirection of the anorthosite xenoliths share a common mean with the directions of their host diabase. Together with thermal conduction modeling, these data are consistent with the interpretation that the anorthosite xenoliths within the Beaver River diabase acquired remanence during cooling with the diabase after their hypabyssal emplacement. A total of  xxx specimens from xx sites were used for paleointensity experiments. The locations of the xx sites are shown in Fig. 1 and the corresponding site mean palaeomagnetic directions from \citeA{Zhang2021a} are shown in Fig. 2(b).




\begin{figure}
\centering
\noindent\includegraphics[width=4.75 in]{Figure/Field_photo.pdf}
\caption{\footnotesize{Field photographs and petrographic images of the Beaver River diabase and the anorthosite xenoliths within it. (A) Centimeter-sized plagioclase megacrysts in the diabase. (B) Rounded anorthosite xenolith with a diameter of $\sim$7 meters fully enclosed within the diabase. (C) Exposure of a giant Carlton Peak anorthosite with a diameter $>$100 m. (D) 27.5 m diameter anorthosite xenolith sampled as paleomagnetic site AX16 and geochronology sample MS99033. (E) Cross polarized (XPL) image of the subophitic texture of diabase at site BD2 (pyroxene partially enclosing plagioclase). (F) XPL image of anorthosite geochronology sample MS99033. Plagioclase crystals exhibit both granoblastic texture and interlocking lath fabrics. (G) Backscattered electron (BSE) image of a large Fe-Ti oxide with titanomagnetite-ilmenite lamellae in Beaver River diabase site BD2. (H) BSE image of micron-sized Fe-Ti oxides exsolved from pyroxene between plagioclase crystals in anorthosite xenolith site AX4. an-plagioclase with $\sim$70\% anorthite; ilm-ilmenite; mag-magnetite; px-pyroxene.}}
\label{fig:Field_photo}
\end{figure}


  
\begin{figure}
\noindent\includegraphics[width=\textwidth]{Field_photo_petro_photo.pdf}
\caption{\small{(A) A typical outcrop of an anorthosite xenolith residing in the Beaver River diabase. (B) A hand sample of anorthosite. The large reflective face at the bottom center is a cleavage plane of a centimeter-size plagioclase crystal. The scale is 9 cm in total. (C) Cross-polarized light image of an anorthosite thin section. The second order birefringence of plagioclase indicates a high Ca content. The closely packed plagioclase crystals show recrystallization texture. (D) High-magnification SEM image of Fe-Ti oxides exsolved from a pyroxene crystal included in a plagioclase crystal. (E) Large, interstitial magnetite-ilmenite intergrowths between two plagioclase crystals. (F) A large Fe-Ti oxide grain with magnetite-ilmenite intergrowths next to a pyroxene in diabase. an: anorthite; px: pyroxene; mag: magnetite; ilm: ilmenite}}
\label{fig:field}
\end{figure}

Most thermally demagnetized anorthosites and alternating field (AF) demagnetized diabase specimens have minimal secondary components and their magnetization is interpreted to be dominated by a primary thermal remanent magnetization. Characteristic magnetizations from both lithologies yield indistinguishable site-mean directions and the virtual geomagnetic poles (VGPs) fall close to the 1095 Ma pole from the synthesized APWP of \cite{Swanson-Hysell2019a}, which is in agreement with the geochronological constraints  (Fig. \ref{fig:pmag}). 

I conducted a comparative study of modified IZZI paleointensity experiments on both the diabase and anorthosite, with a group treated with low-field AF demagnetization after each in-field step and another group without. Paleointensity experiments for most diabase specimens result in non-ideal Arai plots often with poor pTRM checks, making them difficult to interpret for paleointensity estimate (Fig. \ref{fig:pmag}). These results can be interpreted to be associated with observed color change of the diabase after heating, which suggests alterations of the oxides. The experiments conducted on anorthosite samples, on the other hand, had a high success rate of $>$ 50\%. Moreover, the resulting success rate for the anorthosite group with the AF treatment is even higher. The more straight Arai plots are likely due to the AF steps mitigating the exhibition of pTRM tails often associated with non-ideal behavior of multi-domain grains that are demagnetized by the pre-treatment step. However, as the nearly pure anorthosite samples did not display dramatic alterations that are easily detectable with common petrographic techniques, a question rises as why some of the compositionally similar anorthosites passed the paleointensity selection results and some did not. 

\begin{figure}
\noindent\includegraphics[width=\textwidth]{pmag_plot_.pdf}
\caption{\small{(A) Site-mean directions of anorthosite and diabase plotted on an equal-area plot. (B) Calculated VGPs from the anorthosite and diabase plotted in context of a previously synthesized ca. 1.1 Ga Laurentia APWP from the Midcontinent Rift volcanics \citep{Swanson-Hysell2019a}. VGPs from both lithologies fall close to the 1095 Ma pole position. (C) Example Arai plots for diabase. Most diabase and many anorthosite specimens were rejected by selection criteria due to the zigzagging behavior shown in the diabase example plot. (D) Example Arai plots for anorthosite. A typical anorthosite specimen that passes selection shows straight Arai plot with a high paleointensity estimate before cooling rate correction. Note both specimens in (C) and (D) show dominantly single component magnetization in the inset orthogonal plots. The estimated field intensity is not cooling rate corrected.}}
\label{fig:pmag}
\end{figure}

At the IRM, I was seeking the answer to this question with the help of the vibrating sample magnetometer (VSM) systems, including a newly installed Lake Shore VSM which greatly helped improve measurement resolution on samples with weak magnetizations (Fig. \ref{fig:backfield}). Backfield demagnetization experiments were conducted and used to develop coercivity spectra (Fig. \ref{fig:backfield}). The spectra were subsequently modeled to fit for the distributions of different  populations of magnetic particles using similar procedure as \cite{Maxbauer2016a}. Most spectra can be well approximated with models with one component or two components with overlapping coercivity ranges (Fig. \ref{fig:backfield}). The dominance of single-component distributions from the unmixed coercivity spectra likely suggests minimal alterations or formation of secondary magnetic mineral within the samples. By compiling the values of median destructive field (MDF) from all measured specimens and categorizing them in terms of their sister specimens' paleointensity results, I found that the anorthosites samples that pass paleointensity selection criteria have distinctively higher MDF values than those did not (Fig. \ref{fig:backfield}), and all diabase specimens have low MDF values like the non-ideal anorthosites (Fig. \ref{fig:backfield}). The higher MDFs ($\sim$60 mT) of the anorthosites are similar to what is typical of stoichiometric, single-domain magnetite grains which are favored for paleointensity experiments. Furthermore, the low MDFs displayed in other specimens could be linked to a dominant population of large interstitial Fe-Ti oxides that are more likely to form multi-domain grains, which have been observed in both the anorthosite and diabase (Fig. \ref{fig:field}). In addition, preliminary tests on single anorthosite crystals might suggest that not all single crystals are necessarily better for paleointensity research when comparing to bulk samples, given their similar low MDF to the failed diabase and anorthosite specimens. However, more single crystal grains with paired paleointensity data and rock magnetic data may be needed before a general conclusion can be drawn. 

Taken together, the rock magnetism experiment results support my paleointensity selection criteria in that rock specimens that produce straight Arai plots and have MDF similar to stoichiometric single-domain magnetite are preferentially selected. The preliminary paleointensity results yielded consistent estimates. The cooling rate corrected site-mean paleointensity estimate is about 40 $\mu$T, consistent with results from \cite{Sprain2018a} that the Earth's magnetic field strength at surface in the late Mesoproterozoic was close to that of today. 

\begin{figure}
\begin{center}
	\noindent\includegraphics[width=0.5\textwidth]{backfield_example_.pdf}
\end{center}
\caption{\small{(A) Backfield measurement data acquired using a Princeton VSM on an anorthosite specimen. The noise level on coercivity plot is high and a heavy smoothing is needed to make subsequent coercivity modeling interpretable. (B) Backfield measurement data acquired using a Lake Shore VSM on the same specimen as that in (A). The inset coercivity unmixing plot shows that the much smoother initial backfield curve resulted in a tight single-component model fit. (C) Box plot of distribution of the median destructive field of all measured specimens. The anorthosite specimens that pass selection criteria have distinctly higher median destructive field (MDF) than other groups.}}
\label{fig:backfield}
\end{figure}










\section*{Method}

\section*{Results}

\section*{Discussion}

\section*{Conclusion}

\acknowledgments
I thank Dario Bilardello, Peat Solheid, Mike Jackson, Josh Feinberg, and Bruce Moskowitz for their tremendous help of instrumental operations, data interpretations, and research guidance. I thank the IRM for the generous U.S. Visiting Student Fellowship. I thank Nicholas Swanson-Hysell at UC Berkeley for advising this research. Margaret Avery provided field assistance.


\bibliography{YZ_ref}


\end{document}



More Information and Advice:




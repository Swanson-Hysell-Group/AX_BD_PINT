\documentclass[11pt, letterpaper]{article}
\usepackage{xcolor}
\usepackage{textcomp,marvosym}
\usepackage{amsmath,amssymb}
\usepackage[left]{lineno}
\usepackage{changepage}
\usepackage{rotating}
\usepackage{natbib}
\usepackage{setspace}
\usepackage{fancyhdr}
\usepackage{graphicx}
\usepackage{sidecap}
\usepackage[aboveskip=1pt,labelfont=bf,labelsep=period,justification=raggedright,singlelinecheck=off]{caption}
\usepackage{titlesec}
\usepackage{url}

\raggedright
\textwidth = 6.5 in
\textheight = 8.25 in
\oddsidemargin = 0.0 in
\evensidemargin = 0.0 in
\topmargin = 0.0 in
\headheight = .5 in
\headsep = 0.5 in
\parskip = 0.2 in
\parindent = 0.2in
\titlespacing\section{0pt}{12pt plus 4pt minus 2pt}{0pt plus 2pt minus 2pt}

\thispagestyle{myheadings}
\thispagestyle{fancy}
%\fancyhf{}
%\lhead{\Large \textbf {Summer School Scholarship Application} \noindent\small\textit{Yiming Zhang, UC Berkeley}}
\lhead{\noindent\small{Yiming Zhang \\ PhD candidate \\ 307 McCone Hall, Berkeley, CA 94720 \\ yimingzhang@berkeley.edu}}
%\rhead{\thepage}

%\author{Yiming Zhang}
%\date{University of California, Berkeley}
\begin{document}

%\maketitle
\begin{flushleft}
 
\hfill 2022
%{\noindent\textit {Yiming Zhang, UC Berkeley}}

Dear \textit{PNAS} editors,

We appreciate the positive assessment that our manuscript received through review. We are also grateful for the suggestions made by the reviewers that have guided revisions. The implementation of these revisions is detailed below and we have attached a track changes PDF with the submission of the revised manuscript. All reviewer's comments below are in italics and our responses are in regular fonts. 
 
 \textbf{\Large Editor Remarks to Author:}

\textit{This manuscript has now been favorably reviewed by two experts in the field. Both suggested minor to moderate revisions which will improve the manuscript. I agree.}

Thank you! We appreciate the positive assessment of this work.

\textbf{\Large Reviewer 1 comments}

\itshape
This well-written manuscript presents new paleointensity estimates for the Mesoproterozoic which are unusually high, providing a compelling argument against a systematic decline in geomagnetic field strength throughout the Neoproterozoic. This is a significant result, since the previously claimed decline in field strength has been used to argue for young inner core nucleation $\sim$500 Ma. These new results convincingly show that there are several substantial changes in field strength throughout the time period 1500 - 500 Ma, and therefore paleointensity trends may not be a good approach for identifying the onset of core solidification.

The authors have clearly explained their approach, justified the interpretation of their results and the reliability of the chosen samples for paleomagnetic analysis. I recommend this study for publication and have included some suggestions and comments below which I hope the authors will find helpful.

\upshape 

Thank you! We appreciate the positive assessment of this work. The comments have helped improve the manuscript. Below are our point-by-point responses.

\itshape
Major comments:

The results presented highlight two particularly interesting points: first, that the geodynamo did not lack power at ~1 Ga at odds with a dwindling thermal dynamo and second that tectonic processes may be responsible for the transitions in field strength around this time. Both of these points could be discussed in more detail, particularly alternative dynamo mechanisms e.g., Mg and Si precipitation and the likely efficiency of such mechanisms to generate the observed intensities observed here. Additionally, the role of tectonic processes in driving transitions in field strength is only mentioned at the end of the manuscript, it would be useful to describe the influence of tectonics on CMB heat flux in more detail earlier in the paper, and in particular any constrains from previous studies on how large these transitions in heat flux (and therefore field strength) are likely to be.
\upshape

These are good points. We have added a sentence to the introduction that briefly describes that there are proposals for additional power sources to sustain the geomagnetic field including exsolution of Mg and Si-bearing minerals at the core-mantle boundary. We think that the discussion of plate tectonic driven modulation to field strength is better positioned within the discussion than the introduction. 

%Studies have proposed various additional power sources that could have existed regardless of the inner core that could help drive an active geodynamo. \cite{Driscoll2014a} proposed that additional radioactive energy could have been supplied to the core by radioactive isotopes such as $^{40}$K; \cite{Mittal2020a} suggested that the additional chemical flux in the core produced by the precipitation of light elements could have helped maintain a dynamo, and studies such as \cite{Crowley2012a} explored theoretically that the core-mantle boundary modulation on the core heat flux which can allow for the maintenance of an active geodynamo without the existence of the inner core. We have added text in the introduction part of the manuscript to echo with our discussion. 

\itshape
Figure 6 presents a nice compilation of paleointensity estimates from 1.5 Ga to the present day, including the previously proposed decline in field strength put forward by Bono et al., 2019 which is clearly at odds with the strong intensities which form the main result of this work. Based on all the data now available, it would be interesting to also include a simple statistical test to see if, given the sparsity of data and uncertainties, whether all results to-date can be distinguished from a transitional field oscillating around a constant value? I have never been convinced that the polynomial fit put forward by Bono et al., was statistically robust but it would be nice to quantitatively show it isn't with the new data. Additionally, would it be possible to estimate the timescale over which the transitions between high/low fields occur? Is this consistent with what we know about mantle dynamics and the rate at which subducted material should arrive at the CMB? These questions may be beyond the scope of this paper but may be worth further thought down the line.
\upshape

We agree with the overall take-away that our new data are at odds with the polynomial fit of \cite{Bono2019a}. While statistical treatments of compiled data \cite[e.g.][]{Biggin2015a} are worth revisiting, in the present manuscript we think it is the best to focus on these new paleointensity estimates and their relative strength to highlight this main result. These questions are well-positioned for further analysis down the line as the reviewer writes.


\itshape
Minor comments:
\upshape

\textit{Line 6 - compositional buoyancy due to the ejection of light elements from the growing solid inner core}

The text ``due to the exclusion of light elements'' is added to be more specific about the mechanism behind compositional buoyancy.

\textit{Line 8 - also cite (Biggin et al., 2011) for evidence of a field 3.5 Ga}

Citation added. 

\textit{Line 8-18 - may also be worth including the following paper which has just come out (Pozzo et al., 2022). Aside from the age of the inner core, the paleomagnetic record may also allow us to determine the mechanism that drove the dynamo prior to inner core nucleation. It would be good to add a couple of sentences discussing the paradox that if the inner core is young and the thermal conductivity is high thermal convection will be inhibited and unable to drive a dynamo prior to nucleation. In this case other mechanisms need to be considered e.g., Mg or Si precipitation (Badro et al., 2016; Hirose et al., 2016; O'Rourke \& Stevenson, 2016)}

Pozzo et al. 2022 is now cited.

The sentence ``The possibility of late inner core nucleation has motivated proposals of novel power sources to sustain the geomagnetic field through early Earth history including precipitation of light-element minerals such as MgO \citep{Badro2016a, ORourke2016a, ORourke2016b} and SiO$_2$ \citep{Mittal2020a} at the core-mantle boundary.'' has been added which refers to the reviewer's point here.

\textit{Figure 1 - in this context I think providing a simplified geological map is fine, however it would be beneficial to include slightly more information in the legend. E.g., rather than just referring to them as intrusive igneous rocks/extrusive volcanic rocks (also being pedantic, both are igneous!), can you list the range of lithologies denoted by each colour? Similarly, for Pesonen and Halls include the lithologies. Also colour bar with ages is currently over the map, it should be moved to the side.}

In order to provide more details on lithology, we have added descriptive text to the legend associated with each study. At this scale of a geologic map, making further division of the polygons beyond the extrusive vs intrusive categorization would not be legible.

We have moved the color bar so that it does not overlap with any map units.

\textit{Line 57-59 - why does the lack of extension allow these rocks to be preserved?}

Were the Midcontinent Rift extension to have continued, it is very likely that it would have separated Laurentia into two, exposing the rift rocks at a continental margin setting, where subsequent collisions and associated orogenesis would have metamorphosed or completely eroded away the rocks through exhumation. However, the extension failed thereby preserving such rocks in the interior of Laurentia.

We have added ``far from the continental margin and subsequent orogenesis" to succinctly make this point.

\textit{Line 97 - in what way is this rock type unique?}

The paragraph following line 97 explains the distinct petrologic and magnetic properties of the anorthosite xenoliths. To avoid overstating we changed this sentence to ``In this study, we target the high-purity anorthosite xenoliths of the Beaver River diabase in the Midcontinent Rift." 

\textit{Line 99-104 - the wording here could be clearer. Because plagioclase isn't Fe-bearing it cannot alter to form Fe-oxides, whereas minerals like olivine and pyroxenes are Fe-bearing so more likely to alter in a way that influenced the magnetic mineralogy.}

We have split the sentence into two: ``They are attractive targets for paleomagnetic study as plagioclase crystals can protect magnetic inclusions from alteration. In addition, the alteration of the plagioclase crystals does not readily result in the formation of secondary iron oxides in contrast with Fe-silicate minerals such as olivine and pyroxene."

\textit{Line 107 - regarding the xenolith cooling, can you clarify whether they were partially molten or completely solid when they were brought to the near-surface? And specify the temperatures they cooled from, the phases acquiring magnetization and the Curie/blocking temperatures of these phases.}

We have clarified these points to describe that the anorthosites were heated to tholeiitic magma temperatures below the melting point of plagioclase. We also clarify how these means that they cooled from temperatures above the Curie temperature of the magnetite remanence carriers. 

\textit{Line 109 - missing space and only need degree symbol once 22 {plus minus} 2{degree sign}}

Text changed to ``22 $\pm$ 2\textdegree."

\textit{Line 119-122 - this seems a little out-of-place here to conclude the Mesoproterozoic geodynamo must have been strong before presenting paleointensity results.}

We removed this sentence. 

\textit{Line 123-145 - this is a little repetitive with the introduction and I think could be combined with the previous section. When discussing geological setting and rock magnetism, respectively.}

We have integrated this section into the Introduction to avoid repetition. 

\textit{Line 188 - 189 - can you clarify whether remanence anisotropy has also been measured by applying ARMs and demagnetizing them on a 2G? If possible, it would be good to quantify the degree of anisotropy (or lack thereof) using this method for comparison with other anisotropy studies.}

We did not measure remanence anisotropy by the AARM method which would require experiments on distinct sister specimens. Given that it is a metric associated with the specimen on which the paleointensity experiment is being conducted, we prefer to use the gamma statistic which is consistent  with a low degree of TRM anisotropy of our specimens. Note that directional data also show the same mean directions acquired by the anorthosite xenoliths and host diabase consistent with minimal remanence anisotropy.

We have changed the sentence to ``The anorthosite xenoliths show low anisotropy of thermal remanent magnetization (TRM) acquisition and can acquire TRM linearly within the range of relevant field strengths."

\textit{Figure 2 - the QDM maps for the Duluth anorthosite look significantly stronger than for Beaver River because of the choice of Bz scale range. Is it possible to make them the same so that it's a more direct comparison between the two?}

We have decreased the range of the scale for the magnetic maps for the Beaver River anorthosite xenoliths to increase contrast and make the sources more visible. However, bringing all maps to the same scale will either diminish the dipoles in the map of the Beaver River anorthosite xenolith or saturate the color scales in the Duluth anorthosite map. For clarification, we noted in the figure caption that the two maps have Bz scales that are an order of magnitude different.

\textit{Figure 3 - many of the pTRM checks appear to have failed. Can you include some discussion of the degree of similarity between the first and second in-field step required for each pTRM check to pass? Similarly, how much sagging is allowed before a specimen fails? AX16-9a for example, looks like it shows a small amount of sagging - how much more would be tolerated before this specimen would fail? Also include in the figure caption a description of what the dashed blue/yellow lines represent.}

We choose to use the SCAT parameter of \cite{Tauxe2004a} which uses a threshold value to indicate whether the data points (including the pTRM checks and tail checks) in the chosen segment of the Arai plot are too scattered. The combination of the SCAT parameter and the FRAC parameter (which is the fraction of remanence used for determining paleointensity estimate) determines the amount of sagging allowed.

We have added ``Dashed red and blue lines show bounding regions associated with the \textit{SCAT} statistic \citep{Shaar2013a}." to the figure caption.

\textit{Figure 4A - are all uncertainties 1 standard deviation?}

Yes. We edited text of ``(black bars with grey one standard deviation uncertainty boxes)" in the caption.

\textit{Figure 4C - what is the uncertainty on the 3 site mean? This should be included on the figure.}

The 1 standard deviation uncertainty is added to Figure 4C as well as Figure S5.

\textit{Line 311-316 - again can you clarify whether the anorthosite xenoliths are partially/fully molten at this temperature? Is there any mixing/mingling with the surrounding host rock that could influence the composition of the magnetic carriers?}

We clarify this point in the introduction.  

\textit{Line 327 - how do you know 500 (degree sign)C is the characteristic blocking temperature for your samples? Or are you just assuming that by the time they have cooled to this temperature all remanence will have been acquired?}

This is a good point. We did not make it clear in the text. 500\textdegree C is an estimated lab unblocking temperature for the anorthosites based on thermal demagnetization of \cite{Zhang2021b} and the zero-field heating steps of this study. An estimated cooling time for the anorthosite from the Curie temperature of magnetite to 500\textdegree C is about 1.5 kyr. This corresponds to a cooling rate of $1.7\times10^{-9}$ $^\circ$C s$^{-1}$. 

\textit{Line 329 - what temperature range was this cooling estimate calculated for? For cooling from 580 to 500 over 1000 years I get a cooling rate of 2.5e-9 C\/s, although I suspect this doesn't have any influence on your cooling rate correction.}

We have changed the text and the calculation to be more specific. Before we were using a less precise and more conservative order of magnitude estimate. With the increase in specificity associated with our cooling model the text now reads: ``From the thermal history model of \cite{Zhang2021b}, we can estimate the duration over which the diabase and anorthosite cooled from the Curie temperature of magnetite ($\sim$580\textdegree C) to the time when they blocked in the majority of their characteristic natural remanence magnetization ($\sim$500\textdegree  C; Fig. 3; \citealp{Zhang2021b}). We find the cooling time to be $\sim$1.5 kyr, which corresponds to a cooling rate of $\sim1.7\times10^{-9}$ $^\circ$C s$^{-1}$."

\textit{Lines 340-352 - are the anorthosites from similar regions in agreement with statistical significance - e.g., can you show they pass a student-T test? Similarly, are those from different regions in agreement\/ different with any statistical significance? It would also be help to state the recovered paleointensities\/ uncertainties in the text here.}

We have added reference to our table S3 and added statement of t-test results within and between regions in the caption for table S3: ``Taking a threshold p value of 0.01, the independent two-sample t-test results show that specimen paleointensity distributions of site AX12 and AX16, and those of site AX11 and AX13 are indistinguishable; region Kennedy Creek have distinguishable distributions from the other two regions; Silver Bay region and Carlton Peak region values are indistinguishable."

\textit{Line 425 - 'is key' should be 'are key'}

`is key' has been changed to `are key'. 

\textit{Line 434-443 - in addition it's unclear whether a core with a high thermal conductivity could drive a thermally convecting dynamo at all. If something like Mg precipitation or Si precipitation at the CMB is at play instead then might not expect to see the decrease in power anyway (in fact this may make a lot more sense in light of your results)}

We agree with the comments and have added additional text to the introduction section regarding the light element precipitation aspect. 

\textit{Lines 465-484 - this section is quite hard to follow, is there a simpler way to describe these transitions? Perhaps they could be labelled on Figure 6?}

To aid the reader to identify the timing of these transitions, we have added more ticks on the x axis (Age [Ma]) of figure 6. This edit should make the figure and related text easier to follow.  

\textit{Lines 526-538 - the implications for a role in tectonic processes in modulating transitions in intensity\/ reversal frequency of the geodynamo is important and feels a little understated given this is the first time it is really mentioned. Can these ideas be introduced sooner in the abstract/introduction to explain why you have recovered such high paleointensities?}

We think that the discussion of plate tectonic driven modulation to field strength is better positioned within the discussion than the introduction. 

\textit{Line 605 - a MAD of $<$20 seems like quite a high cut-off given the apparent lack of anisotropy and stability of the anorthosite samples. Is there a reason why the data have so much scatter?}

Thanks for pointing this out. This MAD selection criteria needed to be updated. The data in fact have much smaller scatter, it is just the selection criteria we implemented were more generous than necessary. A MAD selection of $<$10 would result in the same selection as the highest specimen MAD is around 8. Our table S1 presents these MAD values. 

\textit{Lines 619-620 - DANG on its own doesn't tell you if components are origin trending, did you use MAD $>$ DANG to identify origin-trending components?}

DANG is the Deviation ANGle which is the angle between the free-floating best-fit direction and the direction between data center of mass and the origin of the vector component diagram  \citep{Tauxe2004a}. As a result, DANG is considered to assess whether the component selected is actually trending toward the origin and we consider our usage to be appropriate. 

\clearpage


\textbf{\Large Reviewer 2 comments}

\itshape
This well-written manuscript describes new geomagnetic paleointensity data from the Proterozoic. These data add to the very few existing paleointensity data prior to the mid-Paleozoic that have been used by other authors to infer the onset of inner core nucleation. In particular, recent paleointensity data from the Ediacaran was used to argue for a late inner core. The data presented here are inconsistent with that interpretation, and the authors argue (rightly, I believe) that the currently-available data make it challenging to use paleointensity alone to determine inner core formation. The issue of Earth evolution and inner core nucleation should be of wide interest and is appropriate for publication in PNAS.

The study appears to be well executed and the arguments in the manuscript are mostly well formed. I recommend publication, but have a few mostly minor comments detailed below.
\upshape

Thank you! We appreciate the positive assessment of this work. The comments have helped improve the manuscript. Below are our point-by-point responses.

\textit{General comments:}

\textit{I'm curious why you think the Beaver River magnetite inclusions are apparently not crystallographically controlled within the plagioclase - in contrast to what is observed in other intrusions. Do you think the magnetite was formed via a different process or at a different temperature? Related, I think at least a brief comment/discussion on the likely temperature of magnetite formation is warranted. (i.e., formed above/below the Curie temperature)}

These are good questions that we also wonder about and would like to explore further. \cite{Ageeva2016a, Ageeva2017a, Ageeva2020a, Ageeva2022a,Bian2021a} provides a a series of detailed petrologic and crystallographic investigation into the origin of the oxide needles that exsolve within plagioclase. \cite{Bian2021a} concludes that these oxides form above 600\textdegree C, thus above the Curie temperature of magnetite.

It has been found that partitioning of Fe into plagioclase is enhanced by high anorthite content of the plagioclase, by high oxygen fugacity, and by high silica content of the melt \citep{Phinney1992a, Longhi1976a}. Our Beaver River anorthosite xenoliths do have higher anorthite content ($\sim$70\% An; bytownite) as compared to the plagioclase of the Duluth Complex ($\sim$60\% An; labradorite). We suspect that the anorthosite xenoliths form during the early stage of the magmatic evolution of the Beaver River diabase when anorthite content, and likely the oxygen fugacity and silica content in the magma were high, leading to more partition of Fe into plagioclase crystal lattice, and minor amount of Fe to exsolve and form oxide needles (although those that did could still be crystallographically controlled), instead of forming abundant, large Fe-oxide exsolutions like those in the Duluth anorthosite. 

While discussion regarding the details about the petrologic origin of the Fe-Ti oxides is beyond the scope of this manuscript, we added a reference to \cite{Bian2021a} in our text of section ``Petrography and magnetic imaging of anorthosite xenoliths" to help clarify the formation of these oxides are above the Curie temperature of magnetite.

\textit{Line 14-15: ``these values" is a little vague. Are you referring just to conductivity, or something else too?}

The text has been changed to ``While some estimated thermal conductivity values are consistent with an inner core age $>$3 Ga..."

\textit{Lines 101-102: For clarity, would suggest changing to ``...from alteration. Alteration of the plagioclase crystals themselves does not result in ..."}

This is a good suggestion. This sentence has been split into two: `` They are attractive targets for paleomagnetic study as plagioclase crystals can protect magnetic inclusions from alteration. In addition, the alteration of the plagioclase crystals does not readily result in the formation of secondary iron oxides in contrast with Fe-silicate minerals such as olivine and pyroxene."

\textit{Line 143: ``unblock" should be ``unblocks"}

``Unblock" has been changed to ``unblocks".

\textit{Line 151-153. If you're going to say that these crystals ``often" contain oxide inclusions, you should probably cite more than one example.}

Additional references to \cite{Wenk2011a, Ageeva2016a} have been added. 

\textit{Lines 176-181 and Fig. 2: For the PNAS audience, I don't think most people will be able to ``see" this magnetization rotation in Figure 2. Even for those who could recognize and interpret the dipole signature, the relevant panels in Figure 2 are a bit hard to make out. Can you enlarge a region with one or two dipoles so it is easier to see the change/deviation from the applied field direction?}

This is a good point. We added enlarged representative magnetic dipoles maps from both samples and the insets are now included in Fig. 2. 

We hope our caption  ``The magnetic images show that remanent magnetizations of Beaver River anorthosite (i.e. the individual dipoles visible with paired red +B$_z$ and blue -B$_z$ lobes) align well with the first applied field direction (E) and then rotated to align with the second applied field direction indicating minimal anisotropic behavior (G)" describes the rotation of the dipoles in detail.

\textit{Figure 5: I think it's a bit deceptive to just include the smoothed fit and not show what are actually quite noisy data (based on what's in your Jupyter notebook). I think the interpretation is fine, but it would be nice to show the original data - either here or in the supplementary materials. (Also, the font size is quite small and difficult to read.)}

The three example specimen shown in this figure actually have quite nice data that are not very noisy. To make the main figure clear and highlight the difference in peak coercivities, we decide to keep the presentation of this figure as is. The font sizes have been increased. 
An additional supplementary figure S6 which plots the fitted coercivity curve and the smoothed data points has been added. 

\textit{Lines 268-271: Also a slightly misleading statement. Passing samples ``can" have peak coercivity distributions around 80 mT, but based on your box plot (Fig5), only one sample has an MDF >80 mT, one has an MDF of about 70 mT, and the rest are <60 mT.}

The sentence has been changed to more fully describe the result of higher coercivity for these samples ``Single-component fits for coercivity spectra \citep{Maxbauer2016a} show that anorthosites having successful paleointensity results can contain magnetic grain populations that have higher coercivities. For these successful samples, median destructive destructive (MDF) field values associated with back-field demagnetization experiments range from 44 to 144 mT with a median of 53 mT (Fig. 5). In contrast, other anorthosite and diabase specimens tend to have lower peak coercivities (MDF range of 18-35 mT with a median of 22 mT; Fig. 5)."

\textit{Line 328: cooling time on the order of $\sim$1 kyr - That's not what is shown in Figure S3. There, you show 10 kyr.}

Thanks for catching this inconsistency. We have updated the text and Fig. S3 so that they are consistent. The text now reads ``From the thermal history model of \cite{Zhang2021b}, we can estimate the duration over which the diabase and anorthosite cooled from the Curie temperature of magnetite ($\sim$580\textdegree C) to the time when they blocked in the majority of their characteristic natural remanence magnetization ($\sim$500\textdegree  C; Fig. 3; \citealp{Zhang2021b}). We find the cooling time to be on the order of $\sim$1.5 kyr, which corresponds to a cooling rate of $\sim1.7\times10^{-9}$ $^\circ$C s$^{-1}$. In contrast, the lab cooling rate is much faster through the same temperature interval with an estimated cooling rate of $\sim1.3\times10^{-1}$ $^\circ$C s$^{-1}$. The significant cooling rate difference leads to a predicted $\sim$33\% overestimate of true ancient field following the model of \citealp{Halgedahl1980a} (Fig. S3). This estimate on cooling rate effect is similar to the value of $\sim$30\% overestimate derived from the model of ref. \citealp{Nagy2021a}. We therefore correct our paleointensity results by a factor of 0.75."

\textit{Lines 331-333. Both methods of assessing cooling rate bias assume uniaxial SD, yet the zig-zag behavior shown in at least 2 out of 3 of your passing Arai plots (Fig. 3) suggest something larger than SD. How might that affect your cooling rate calculations?}

This point is a good one given that PSD and MD grains may be less biased toward recording higher paleointensities than SD grains associated with to slow cooling. We have added text ``Remanence held by vortex state (pseudo-single domain) and multidomain grains is not as biased by cooling rate \citep{Biggin2013a}. The potential for some remanence to be held by these grains, as suggested by slightly zigzagging Arai plots (Fig. 3), could mean that this factor is an over-correction and true paleointensity values could be slightly higher than those reported here. The cooling rate-corrected specimen paleointensity estimates together with specimen- and site-level means are shown in Fig. 4." to the related paragraph. 

\textit{Lines 356-360: Do you think any of the intensity difference between your samples and the PADM2M model can be explained by the fact that the model only reflects the axial dipole contribution to the field? Also, I believe the necessarily smoothing in most of these models will likely underestimate field variability. I don't know if that will necessarily lead to a lower average, but it probably underrepresents the high field values.}

This point is an interesting one as local high flux patches are not represented in an axial dipole mode. While we think the comparison to PADM2M is valuable, this aspect of the comparison is what also motivated us to compare to the Cenozoic paleointensity database (where the values from this study are in the 97 percentile). This comparison is described in the text and illustrated in Figure S5.

\textit{Figure 6: Some of the pale colors are very hard to see.}

We have decreased transparency on the data points from the compilation. Now they can be seen more clearly. 

\textit{Supplementary material: Assuming this material will not be reformatted, please consider enlarging the font size in the figure captions and within many of the figures. It is quite difficult to read without constantly zooming in and out.}

The fonts have all been enlarged.

\textit{Figure S1. What does the yellow dashed line represent? Please explain red vs. blue and ``A" vs ``B" sites.}

The caption has been updated to clairfy the symbology.

\textit{Figure S2. Can you indicate which of these specimens come from ``passing" samples and which are from ``failing" samples? Also, I'm pretty sure the ZFC$>$FC being an indication of MD material has been demonstrated to be unreliable. (Sorry I can't find the relevant reference.)}

Labels for whether the sister specimen to the rock chips used for MPMS measurements have been added. Figure caption has been updated with the following sentence: ``Pass" or ``Fail" represents whether their sister specimens passed our paleointensity selection criteria or not. We are not aware of the demonstration of ZFC$>$FC as being unreliable. The papers of Carter-Stiglitz et al. nicely demonstrate the effect for magnetite and titanomagnetite. Jackson and Moskowitz (2021) cite this effect as giving insight into MD behavior which indicates that these experts still consider it to be a reliable diagnostic.  

\textit{Data availability: The link to the MagIC database did not work, so I was unable to verify how much of the data were uploaded. It's unclear if it is just the paleointensity data or if it includes all the related rock magnetic data.}

The link has been updated: \url{https://earthref.org/MagIC/19462/8d3c2258-11ae-4830-b99f-3f6b02eceb7e}. 


Sincerely,

Yiming Zhang\\

PhD candidate of Earth and Planetary Science\\

University of California, Berkeley

\bibliographystyle{gsabull}
\bibliography{YZ_ref}

\end{flushleft}
\end{document}
\documentclass[11pt, letterpaper]{article}
\usepackage{xcolor}
\usepackage{textcomp,marvosym}
\usepackage{amsmath,amssymb}
\usepackage[left]{lineno}
\usepackage{changepage}
\usepackage{rotating}
\usepackage{natbib}
\usepackage{setspace}
\usepackage{fancyhdr}
\usepackage{graphicx}
\usepackage{sidecap}
\usepackage[aboveskip=1pt,labelfont=bf,labelsep=period,justification=raggedright,singlelinecheck=off]{caption}
\usepackage{titlesec}

\raggedright
\textwidth = 6.5 in
\textheight = 8.25 in
\oddsidemargin = 0.0 in
\evensidemargin = 0.0 in
\topmargin = 0.0 in
\headheight = .5 in
\headsep = 0.5 in
\parskip = 0.1 in
\parindent = 0.2in
\titlespacing\section{0pt}{12pt plus 4pt minus 2pt}{0pt plus 2pt minus 2pt}

\thispagestyle{myheadings}
\thispagestyle{fancy}
\fancyhf{}
%\lhead{\Large \textbf {Summer School Scholarship Application} \noindent\small\textit{Yiming Zhang, UC Berkeley}}
\lhead{\noindent\small{Yiming Zhang \\ PhD candidate \\ 307 McCone Hall, Berkeley, CA 94720 \\ yimingzhang@berkeley.edu}}
%\rhead{\thepage}

%\author{Yiming Zhang}
%\date{University of California, Berkeley}
\begin{document}

%\maketitle
\begin{flushleft}
 
\hfill \today
%{\noindent\textit {Yiming Zhang, UC Berkeley}}

Dear \textit{PNAS} editors,

This letter accompanies our submission of a manuscript entitled ``High geomagnetic field intensity recorded by anorthosite xenoliths requires a strongly powered late Mesoproterozoic geodynamo." 

Earth's long-term thermal evolution and the timing of inner core nucleation is currently very difficult to solve from a theoretical perspective as there continue to be conflicting estimates from the mineral physics community on relevant thermal conductivity values. While some in the paleomagnetic community have embraced a late timing of inner core nucleation associated with high thermal conductivity values, such an interpretation is reliant on the interpretation that weak fields ca. 565 Ma capture a near demise of a thermally driven dynamo that was followed by inner core nucleation. Given the sparse paleointensity database of the Proterozoic, acquiring new high-fidelity observational records of ancient magnetic field intensity from the remanent magnetization of rocks is crucial for constraining the long-term evolution of Earth's core from an observation perspective. However, robust estimates of ancient field strengths are often difficult to recover due to alteration or non-ideal rock magnetic behavior that is found in many igneous rocks. 

In this study, we target a quite unique rock type which are lower crustal anorthosite cumulates that were brought to the near surface where they cooled and acquired their magnetization 1.1 billion years ago. In contrast to the diabase that hosts the xenoliths which exhibits non-ideal behavior, the anorthosites are high fidelity paleointensity recorders. These anorthosite xenoliths are different than those typical of layered mafic intrusions as they formed deeper in the crust, but then acquired their magnetization near to the surface. We are able to use $\mu$m-scale magnetic maps generated with a quantum diamond microscope to show that in contrast to typical anorthosites with exsolved magnetite needles there is minimal grain-scale anisotropy associated with the small magnetite hosted within the plagioclase crystals. The data that emerges from this unique archive shows that the geomagnetic field was quite strong 1.1 billion years ago --- stronger than any time in the past 2 million years and rivaling some of the strongest dipole moments of the entire database. Rather than there being a progressive monotonic decay of Earth's dynamo strength through the Proterozoic Eon, there was a strongly powered dynamo in the late Mesoproterozoic.

Multiple observed transitions from weak to strong paleointensity values in the Phanerozoic and the Proterozoic cannot all be the minimum prior to the beginning of inner core nucleation. Instead, they likely are the result of plate tectonics modulating core-mantle heat flow patterns in the Proterozoic as in the Phanerozoic (giving rise to intervals such as the mid-Paleozoic dipole low). We are excited about the results presented in this manuscript and believe that the readers of \textit{PNAS} will find them compelling and a significant advance.

Sincerely,

Yiming Zhang\\

% PhD student of Earth and Planetary Science\\

% University of California, Berkeley


% In this contribution, we develop high-quality paleointensity data with the best site-level standard deviation of 0.04 $\mu$T from well-preserved anorthosite xenoliths of the Beaver River diabase which were rapidly emplaced during the ca. 1.1 Ga North America Midcontinent Rift. The primary interpretation for the high paleointensity values gains strong support for there being a strongly powered geodynamo during the late Mesoproterozoic. These observations agree with previous results from Midcontinent Rift volcanic rocks and support such a strong field could have lasted at least 14 Myr. Together, paleointensity records from the Midcontinent Rift are inconsistent with there being a progressive monotonic decay of Earth's dynamo strength through the Proterozoic Eon.
 
%   Our updated paleointensity data compilation show multiple observed paleointensity transitions from weak to strong in the Phanerozoic and the Proterozoic. Given that these records cannot all be the minimum prior to the beginning of inner core nucleation, they instead could indicate that processes such as plate tectonics could have modulated the core-mantle heat flow pattern in the Proterozoic. Regardless of when the initiation of inner core nucleation occurred, the observation of large variability in paleointensity in the Precambrian may present challenges in detecting the increase in surface geomagnetic field strength predicted by numerical models to have happened at the onset of inner core crystallization. 
 
% 





% \clearpage

% \textbf{FAIR data availability:}\\
% The measurement level paleomagnetic data are in the FAIR data repository MagIC and a link within the manuscript makes these data available for reviewers.

% \textbf{Suggested editor reviewers:}\

% Lisa Tauxe (ltauxe\@ucsd.edu)\

% \textit{expertise in paleointensity and particularly in the behavior of the ancient geomagnetic field}

% Joshua Feinberg (feinberg\@umn.edu)\

% \textit{expertise in rock magnetism and paleomagnetism}

% Andrew Biggin (A.Biggin\@liverpool.ac.uk)\

% \textit{expertise in paleomagnetism and particularly in the long term evolution of Earth interior}

% Peter Selkin (paselkin\@uw.edu)\

% \textit{expertise in paleomagnetism and particularly in the  magnetic properties of magnetic minerals in feldpsar}

% Peter Driscoll (pdriscoll\@carnegiescience.edu)\

% \textit{expertise in numerical modeling of Earth's thermal and magnetic evolution}
% \textit{}
\end{flushleft}
\end{document}
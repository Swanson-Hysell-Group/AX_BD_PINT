\documentclass[11pt, letterpaper]{article}
\usepackage{xcolor}
\usepackage{textcomp,marvosym}
\usepackage{amsmath,amssymb}
\usepackage[left]{lineno}
\usepackage{changepage}
\usepackage{rotating}
\usepackage{natbib}
\usepackage{setspace}
\usepackage{fancyhdr}
\usepackage{graphicx}
\usepackage{sidecap}
\usepackage[aboveskip=1pt,labelfont=bf,labelsep=period,justification=raggedright,singlelinecheck=off]{caption}
\usepackage{titlesec}

\raggedright
\textwidth = 6.5 in
\textheight = 8.25 in
\oddsidemargin = 0.0 in
\evensidemargin = 0.0 in
\topmargin = 0.0 in
\headheight = .5 in
\headsep = 0.5 in
\parskip = 0.1 in
\parindent = 0.2in
\titlespacing\section{0pt}{12pt plus 4pt minus 2pt}{0pt plus 2pt minus 2pt}

\thispagestyle{myheadings}
\thispagestyle{fancy}
\fancyhf{}
%\lhead{\Large \textbf {Summer School Scholarship Application} \noindent\small\textit{Yiming Zhang, UC Berkeley}}
\lhead{\noindent\small{Yiming Zhang \\ PhD candidate \\ 307 McCone Hall, Berkeley, CA 94720 \\ yimingzhang@berkeley.edu}}
%\rhead{\thepage}

%\author{Yiming Zhang}
%\date{University of California, Berkeley}
\begin{document}

%\maketitle
\begin{flushleft}
 
\hfill 2022
%{\noindent\textit {Yiming Zhang, UC Berkeley}}

Dear \textit{Proceedings of the National Academy of Sciences} editors,

 This letter accompanies our submission of a manuscript entitled ``High geomagnetic field intensity recorded by anorthosite xenoliths requires a strongly powered late Mesoproterozoic geodynamo". 
 
 Acquiring high-fidelity observational records of ancient magnetic field intensity from the remanent magnetization of rocks is crucial for constraining the long-term evolution of Earth's core. However, robust estimates of ancient field strengths are often difficult to recover due to alteration or non-ideal rock magnetic behavior. In this contribution, we develop high-quality paleointensity data with the best site-level standard deviation of 0.04 $\mu$T from well-preserved anorthosite xenoliths of the Beaver River diabase which were rapidly emplaced during the ca. 1.1 Ga North America Midcontinent Rift. The primary interpretation for the high paleointensity values gains strong support for there being a strongly powered geodynamo during the late Mesoproterozoic. These observations agree with previous results from Midcontinent Rift volcanic rocks and support such a strong field could have lasted at least 14 Myr. Together, paleointensity records from the Midcontinent Rift are inconsistent with there being a progressive monotonic decay of Earth's dynamo strength through the Proterozoic Eon.
 
  Our updated paleointensity data compilation show multiple observed paleointensity transitions from weak to strong in the Phanerozoic and the Proterozoic. Given that these records cannot all be the minimum prior to the beginning of inner core nucleation, they instead could indicate that processes such as plate tectonics could have modulated the core-mantle heat flow pattern in the Proterozoic. Regardless of when the initiation of inner core nucleation occurred, the observation of large variability in paleointensity in the Precambrian may present challenges in detecting the increase in surface geomagnetic field strength predicted by numerical models to have happened at the onset of inner core crystallization. 
 
We are excited about the results presented in this manuscript and believe that the Earth science community and readers of \textit{Proceedings of the National Academy of Sciences} will find them compelling and a significant advance.



Sincerely,

Yiming Zhang\\

PhD student of Earth and Planetary Science\\

University of California, Berkeley

\clearpage

\textbf{FAIR data availability:}\\
The measurement level paleomagnetic data are in the FAIR data repository MagIC and a link within the manuscript makes these data available for reviewers.

\textbf{Suggested editor reviewers:}\

Lisa Tauxe (ltauxe\@ucsd.edu)\

\textit{expertise in paleointensity and particularly in the behavior of the ancient geomagnetic field}

Joshua Feinberg (feinberg\@umn.edu)\

\textit{expertise in rock magnetism and paleomagnetism}

Andrew Biggin (A.Biggin\@liverpool.ac.uk)\

\textit{expertise in paleomagnetism and particularly in the long term evolution of Earth interior}

Peter Selkin (paselkin\@uw.edu)\

\textit{expertise in paleomagnetism and particularly in the  magnetic properties of magnetic minerals in feldpsar}

Peter Driscoll (pdriscoll\@carnegiescience.edu)\

\textit{expertise in numerical modeling of Earth's thermal and magnetic evolution}
\textit{}
\end{flushleft}
\end{document}